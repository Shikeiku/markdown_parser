% Packages
%% Math packages
\usepackage{amsmath, amssymb}
%% Framing for boxes
\usepackage{tikz}
\usepackage[framemethod=TikZ]{mdframed}
\usetikzlibrary{intersections}
%% Listings package for importing code, important!
\usepackage{listings}
%% Formatting
%%% Headers and sections
\usepackage{fancyhdr}
\usepackage{titlesec}
%%% Spacing
\usepackage{setspace}
\usepackage{parskip}
%%% Colors and figures
\usepackage{color}
\usepackage{graphicx}
\usepackage{import}
\usepackage{pdfpages}
\usepackage{transparent}
\usepackage{svg}
\usepackage{babel}
\graphicspath{{./figures/}}
% Misc.
\usepackage{lipsum}
\usepackage{xifthen}
% \numberwithin{equation}{section}

\usepackage{etoolbox}
\usepackage{hyperref}



\hypersetup{
    % linktocpage,
    colorlinks,
    linkcolor={red!50!black},
    citecolor={blue!50!black},
    urlcolor={blue!80!black}
}


\usepackage[%
    a4paper,
%   includeheadfoot,
    head=\baselineskip,  % distance from bottom of header to block of text aka \headsep e.g. \baselineskip
    foot=2.3cm,  % distance from top of footer to block of text aka \footskip
    headheight=5pt,     % height for the header block (no equivalent for footer)
%   heightrounded,       % ensure an integer number of lines
    marginparwidth=2cm,  % right marginal note width
    marginparsep=2mm,    % distance from text block to marginal note box
%   height=\textheight,  % height of the text block
%   width=\textwidth,    % width of the text block
    top=2.5cm,           % distance of the text block from the top of the page
    bottom=3cm,
    left=2.5cm,
    right=2.5cm,
%    showframe,           % show the main blocks
%    verbose,             % show the values of the parameters in the log file
]{geometry}

% memoir things
% \setpnumwidth{10em} % changes the dots after the sections listings in the
%% table of contents 
\renewcommand*{\cftchaptername}{}
% \renewcommand{\chapternumberline}{.}
\renewcommand*{\cftchapteraftersnum}{.}

% toc title
\renewcommand{\printtoctitle}[1]{\par\centering\Large\bfseries\scshape #1 \par}
\renewcommand{\afterchaptertitle}{\vspace{0.5em}}

	\mdfdefinestyle{titlebox}{frametitleaboveskip=-\ht\strutbox,
 frametitle={\colorbox{white}{\space \normalfont Chapter \thechapter \space}},
    frametitlealignment=\lfill, skipabove=\baselineskip, skipbelow=0.5em, linewidth=0.5pt,
  innerleftmargin=3ex, innerrightmargin=3ex,
innertopmargin=0em, innerbottommargin=1em, innermargin =+0cm, outermargin =+0cm}
	\newmdenv[style=titlebox, nobreak=true,
  topline=true,
  bottomline=true,
  leftline=true,
  rightline=true]{titlebox}
% Making my own chapterstyle


\makechapterstyle{mike-section}{
  \renewcommand*{\printchaptername}{}
  \renewcommand*{\chapternamenum}{}
  \renewcommand*{\chapnumfont}{\chaptitlefont}
  \renewcommand*{\printchapternum}{}%{\chapnumfont \thechapter\space}
  \renewcommand*{\afterchapternum}{}
  \setlength{\beforechapskip}{\baselineskip}
  \setlength{\afterchapskip}{\baselineskip}

  \renewcommand*{\clearforchapter}{}
  \renewcommand*{\chapterheadstart}{\begin{titlebox}[]}
  \renewcommand*{\printchaptertitle}[1]{\bfseries\huge\centering #1}
  \renewcommand*{\afterchaptertitle}{\end{titlebox}}
}

% \setsecheadstyle{\normalfont\centering\MakeUppercase}
% \setaftersecskip{-1em}

\newcommand*{\quoter}[1]{\makebox[\textwidth][c]{\begin{minipage}{0.9\linewidth}\itshape"#1"\end{minipage}}}

% Section

%Commands for math notes
%% General commands
\newcommand*{\pb}{\pagebreak}%
%% Math commands
%%% QED
\newcommand*{\QED}{\hfill\ensuremath{\square}\medskip}%
%%% Number Fields
\newcommand*{\N}{\mathbb{N}}%
\newcommand*{\Z}{\mathbb{Z}}%
\newcommand*{\Q}{\mathbb{Q}}%
\newcommand*{\R}{\mathbb{R}}%
\newcommand*{\C}{\mathbb{C}}%




% Figures preamble for math
	%% Global figure width
	\setkeys{Gin}{width=\textwidth}
	%% This is not used any more, but can be used for including PDFs with separate text files
	% \newcommand{\incfig}[1]{%
	%     \def\svgwidth{\columnwidth}
	%     \import{./figures/}{#1.pdf_tex}
	% }
	% \makeatother
	%% Used for including figures I export from affinity designer
	\newcommand{\incsvg}[1]{%
	    \includegraphics{#1.pdf}
	}

	\makeatother




%% Titles and headers of notes
%	% Title
%	%% Title macro that is printed with the ~/scripts/compyler.py script
%	\makeatother
%	\def\@note{}%
%	\newcommand{\note}[3]{
%	    \ifthenelse{\isempty{#3}}{%
%		\def\@note{Note #1}%
%	    }{%
%		\def\@note{Note #1: #3}%
%	    }%
%	    \subsection*{\@note}
%	    \marginpar{\small\textsf{\mbox{#2}}}
%	}
%	\makeatletter
%
	% Header/footer
	%% Here the header and footers are defined. Leftmark is the last chapter or section,
	%% and the page is just the pagen\leftmarkumber.


\makeatother
	\pagestyle{fancy}
	\fancyhead[RO,LE]{\thepage} % Right odd,  Left even
\fancyhead[RE,LO]{\textit{\nouppercase{\leftmark}}}          % Right even, Left odd

	\fancyfoot[RO,LE]{}  % Right odd,  Left even
	\fancyfoot[RE,LO]{}          % Right even, Left odd
	\fancyfoot[C]{}          % Right even, Left odd
	% \fancyfoot[C]{\leftmark}     % Center
\makeatletter





% Boxes for programming
	% Styles
	%% Define styles for the mdframed environment here
	% \mdfdefinestyle{normalbox}{skipabove=0pt, skipbelow=1.5ex, linewidth=1pt, innerleftmargin=3ex, innerrightmargin=3ex, innertopmargin=1ex, innermargin =+0cm, outermargin =+0cm}
	% \newmdenv[style=normalbox, nobreak=true]{normalbox}

	% % Environments
	% %% Define environment names here with a counter
	% %%% Theorem
	% \newcounter{theorem}[chapter]
	% \newenvironment{theorem}%
	% {%
	% \medskip\begin{normalbox}\noindent\refstepcounter{theorem}\textbf{theorem} \thetheorem .%
	% }%
	% {%
	% \end{normalbox}%
	% }


	% %%% Definition
	% \newcounter{definition}[chapter]
	% \newenvironment{definition}%
	% {%
	% 	\medskip\begin{normalbox}\noindent\refstepcounter{definition}\textbf{definition} \textbef{\thedefinition} .%
	% }%
	% {%
	% \end{normalbox}%
	% }

	% %%% Corollary
	% \newcounter{corollary}[chapter]
	% \newenvironment{definition}%
	% {%
	% 	\medskip\begin{normalbox}\noindent\refstepcounter{corollary}\textbf{corollary} \textbef{\thecorollary} .%
	% }%
	% {%
	% \end{normalbox}%
	% }



\usepackage{amsthm}
\theoremstyle{definition}
\mdfsetup{skipabove=1em,skipbelow=1em,
  topline=false,
  bottomline=false,
rightline=false,
innertopmargin=-0.5em}
\newmdtheoremenv[nobreak=true]{theorem}{Theorem}[chapter]
\newmdtheoremenv[nobreak=true]{definition}{Definition}[chapter]
\newmdtheoremenv[nobreak=true]{property}{Property}
\newmdtheoremenv[nobreak=true]{consequence}{Consequence}
\newmdtheoremenv[nobreak=true]{lemma}{Lemma}
\newmdtheoremenv[nobreak=true]{proposition}{Proposition}
\newmdtheoremenv[nobreak=true]{law}{Law}
\newmdtheoremenv[nobreak=true]{postulate}{Postulate}
\newtheorem{conclusion}{Conclusion}
\newtheorem{corollary}{Corollary}
\newtheorem{intuition}{Intuition}
\newtheorem*{previously seen}{As previously seen}
\newtheorem*{intermezzo}{Intermezzo}
\newtheorem*{note}{Note}
\newtheorem*{observe}{Observe}
% \newtheorem*{exercise}{Exercise}
\newtheorem*{remark}{Remark}
\newtheorem*{practical}{Practical}
\newtheorem*{problem}{Problem}
\newtheorem*{terminology}{Terminology}
\newtheorem*{application}{Application}
\newtheorem*{ex}{Exercise}
\newtheorem*{question}{Question}

% environments
%% Proof
%% Example
\newenvironment{ex}%
{%
\textit{example}%
}

\newmdenv[
  topline=false,
  bottomline=false,
  rightline=false,
  skipabove=\topsep,
  skipbelow=\topsep
]{siderule}

\newmdenv[
  topline=false,
  bottomline=false,
  leftline=false,
  rightline=true,
  innertopmargin=0.2em,
  skipabove=\topsep,
  skipbelow=\topsep
]{sideruleproof}

\newenvironment{prettyproof}
{
  \begin{sideruleproof} 
\begin{proof}
}
{
\end{proof}
  \end{sideruleproof} 
}


% Todonotes and inline notes in fancy boxes
\usepackage{todonotes}
\usepackage{tcolorbox}

% Make boxes breakable
\tcbuselibrary{breakable}




% Verbetering is correction in Dutch
% Usage: 
% \begin{verbetering}
%     Lorem ipsum dolor sit amet, consetetur sadipscing elitr, sed diam nonumy eirmod
%     tempor invidunt ut labore et dolore magna aliquyam erat, sed diam voluptua. At
%     vero eos et accusam et justo duo dolores et ea rebum. Stet clita kasd gubergren,
%     no sea takimata sanctus est Lorem ipsum dolor sit amet.
% \end{verbetering}
\newenvironment{verbetering}{\begin{tcolorbox}[
    arc=0mm,
    colback=white,
    colframe=green!60!black,
    title=Opmerking,
    fonttitle=\sffamily,
    breakable
]}{\end{tcolorbox}}

% Noot is note in Dutch. Same as 'verbetering' but color of box is different
\newenvironment{noot}[1]{\begin{tcolorbox}[
    arc=0mm,
    colback=white,
    colframe=white!60!black,
    title=#1,
    fonttitle=\sffamily,
    breakable
]}{\end{tcolorbox}}

% Exercise 
% Usage:
% \oefening{5}
% \suboefening{1}
% \suboefening{2}
% \suboefening{3}
% gives
% Oefening 5
%   Oefening 5.1
%   Oefening 5.2
%   Oefening 5.3
\newcommand{\exercise}[1]{%
    \def\@oefening{#1}%
    \subsection*{Exercise #1}
}

\newcommand{\subexercise}[1]{%
    \subsubsection*{Exercise \@oefening.#1}
}


\usepackage{cancel}
\nonzeroparskip
